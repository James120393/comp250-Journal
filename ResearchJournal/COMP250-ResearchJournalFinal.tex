\documentclass{scrartcl}

\usepackage[hidelinks]{hyperref}
\usepackage[none]{hyphenat}

\title{COMP250-Research Journal}
\subtitle{AI and its Future}

\author{1506530}

\begin{document}
	
	
	\maketitle
	\section{Journal}
	
	\abstract{In modern day computers AI are becoming more and more commonplace, but where does it end. This journal was written to explore the future of AI within the world, from what AI is to what its going to be. Exploring this fires up some ethical questions along the way, for example how much control do we allow AI to have? or If an AI becomes sentient does it have rights? these questions are difficult to answer but from the papers that were looked into, they are the questions on some researchers minds.}
	\newline
	\newline
	\textbf{Title: Foreword What is AI? And What Does It Have to Do with Software Engineering? \cite{What}}
	\newline
This paper goes through the basics of what a few different AI's are, and judges them accordingly through heuristic analysis. These heuristics include scope, power, level of code, purpose and knowledge it will obtain. The author provides a reference for "What and AI is" though it is mentioned that there are other definitions out there. The one this author went for was "getting machines to do something that people would agree is intelligent. This is a loose definition, as the term "people" would have to be defined as well, though for this purpose its not required to pin point the meaning of AI. Only that its capabilities are increasing at a rapid rate, one use for AI is stated in the following paper.
	\newline
	\newline
	\newline
	\textbf{Title: The Future of AI in Space \cite{Space}}
	\newline
AI on Mars, and the Moon, that is this papers discussion, the future of space exploration. Humans are limited in their physical and mental capabilities, and therefore rely on technology to achieve what they can not. NASA is one such institute pushing for a more human robotic cooperation for space travel. Their hope is that while humans are still essential to most space faring missions, an AI controlled robot should be used for the more dangerous and challenging tasks as to keep the people safe. This effort requires much work in creating a more powerful efficient AI system that can control multiple robotics ranging from data-gathering scouts to habitat construction. Though the question that is sparked here is that "how much power, freedom and intelligence do we give these AI?". This could pose great risk of the AI going rough and killing many people as humans no longer control their own life support systems.
	\newline
	\newline
	\newline
	\textbf{Title:  Current and future trends in AI \cite{Trends}}
	\newline
The authors in this paper have looked into a brighter side of AI. They discuss AI in three stages embryonic, embedded and embodied AI, and stating that we are currently in the embedded stage where AI is still not "intelligent" but is well on its wy to being such. As well as mentioning that AI is now commonplace within the world of today, where teenagers could not imagine a world without semi-intelligent devices, like Siri in their phones. The authors move onto describe the technology seeds that were created from AI, including Swarm-Bots. They use Swarm-Bots as an example of an AI that could save many lives during natural disasters and preserving Earth by monitoring ecosystems. Though according to their estimations AI will not be advanced enough for this for another 50 years. These kinds of applications for AI shine a light on the better side of computing, though throughout this there will always be a military risk, something thats not mentioned in this paper.
	\newline
	\newline
	\textbf{Title: Evolution, Sociobiology, and the Future of Artificial Intelligence \cite{Evolution}}
	\newline
This paper delves into the different kinds of AI that can be designed and puts them into families, Expert, Autonomous, Cognitive, AI theory and Turing Test AI. The paper then moves onto explaining each and states that when developing AI companies tend not to choose just one family but combine two or three together to achieve their goal. The author conveys that to ensure AI is successful in spreading around the world and getting the highest quality, money and the right application is needed. By right application the author means thats currently AI's spread is limited to what is useful and to increase that spread the AI needs to tap into human tendencies to allow the user to bond with them. Following is a paper that addresses this point.
	\newline
	\newline
	\newline
	\textbf{Title: Future Relations between Humans and Artificial Intelligence: A Stakeholder Opinion Survey in Japan \cite{Japan}}
	\newline
A survey was done in Japan which involved a variety of backgrounds, which includes the general public and people with scientific background. The authors wanted to know what people thought an AI should control and what they thought a human should control, as well as a more symbiotic relationship where both control. The results obtained from this survey were quite interesting: where it was generally agreed that an AI should command driving, disaster prevention and military activities, but personal care was decided that the people should maintain control. Its interesting that people are willing to put their lives in the hands of an AI when they get into a car but not to operate on them, both activities involve risk, would it not be better to allow the precision AI to operate on you as well, as there would be a much lesser chance of them making a mistake. Though the paper does list considerations that must be taken when allowing AI to perform any of these tasks, to sum up the list: governing bodies must be elected to prevent the misuse of the AI systems. 
	\newline
	\newline
	\newline
	\textbf{Title: How to Avoid a Robotic Apocalypse: A Consideration on the Future Developments of AI, Emergent Consciousness, and the Frankenstein Effect \cite{Apocalypse}}
	\newline
Avoiding an AI apocalypse, this paper was the most interesting. It goes through several scenarios of what could or should we do if AI were to become sentient and self-aware. The authors ask several questions relating to if an AI was conscious, should it be afforded ethical and legal rights, or what would the Value system of an AI be? just these questions alone could spark up a huge debate. It is my opinion that all of these questions must be answered before the further development of AI otherwise we could get caught out, and disaster could strike. This paper is another to recognize the need for an international governing body who can dictate the creating and use of AI as to ensure all ethical steps are taken to prevent damage to both the AI system and the people creating it. 

\section{Final Remarks}
To quote a great film, Jurassic Park, Ian Malcolm (played by Jeff Goldblum) states that the: “…scientists were so preoccupied with whether or not they could (bring back dinosaurs), that they didn't stop to think if they should”\cite{Park}. This quote rings heavily with The development of AI as it will never be stopped nor should it stop, but for future development, considerations have to be made, and steps must be taken before we jump into the deep end and create something smarter, faster and more powerful than ourselves.
	
	
	
	\bibliographystyle{ieeetr}
	\bibliography{COMP250-Journal}
	
\end{document}